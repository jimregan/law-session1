\documentclass{beamer} %[compress, blue]
\mode<presentation>

\setbeamercolor{frametitle}{fg=lightred,bg=darkred}
\setbeamercolor{title}{fg=lightred,bg=darkred}
\usepackage[spanish,english]{babel}
\usepackage{ucs}
\usepackage{times}
\usepackage[T1]{fontenc}
\usepackage{tipa} %ŋ character
\usepackage[utf8x]{inputenc}
\usepackage[absolute,overlay]{textpos}
\usepackage{graphicx}
%\usepackage[small,bf]{caption}
%\usepackage{tabularx}
\usepackage{tikz}
\usepackage{url}
\usepackage{gb4e}
\usepackage{linguex}
\usepackage{cgloss4e}
\usepackage{multirow}
\usepackage{ragged2e} 
\usepackage{wasysym}
\usepackage{alltt}

\usetheme{Warsaw}

\definecolor{darkred}{RGB}{153,0,0}
\definecolor{lightred}{RGB}{226,200,200}
\definecolor{ranis}{RGB}{ 128,128,128}

\newcommand\cyrtext[1]{{\fontencoding{T2A}\selectfont #1}}
\newcommand\grktext[1]{{\fontencoding{LGR}\fontfamily{pmt}\selectfont #1}}

\useoutertheme[subsection=false]{smoothbars}

%\usecolortheme{seahorse} %beetle, albatross, fly, default light, seahorse, crane, dove
\usecolortheme[named= darkred]{structure}

\setbeamertemplate{footline}[text line]{} % makes the footer EMPTY
\setbeamertemplate{headline}[text line]{} % header empty

\setbeamersize{text margin left=0.5cm}
\setbeamertemplate{navigation symbols}{}


%\logo{\includegraphics[height=1.6cm]{logoLAW.png}}
\pgfdeclareimage[height=1.6cm]{logo}{logoLAWnotrans.png}

\setlength{\TPHorizModule}{1mm}
\setlength{\TPVertModule}{1mm}

\newcommand{\MyLogoCentred}{
\begin{textblock}{14}(57.2,10.5)
  \pgfuseimage{logo}
\end{textblock}
}

\newcommand{\MyLogoBottomCentred}{
\begin{textblock}{14}(53.5,70)
  \pgfuseimage{logo}
\end{textblock}
}

\newcommand{\MyLogoBottomRight}{
\begin{textblock}{14}(112.2,80.0)
  \pgfuseimage{logo}
\end{textblock}
}

\newcommand{\MyLogo}{
\begin{textblock}{14}(112.2,0.5)
  \pgfuseimage{logo}
\end{textblock}
}

\date{2nd May 2011}
\title{Session 1: Dictionary basics}

\author{Jimmy O'Regan}

\begin{document}


\frame{\titlepage \MyLogoBottomCentred}


\begin{frame}
  \frametitle{Dictionary Basics: Table of Contents}
  \tableofcontents
\end{frame}


\section{Preamble}

\begin{frame}
  \frametitle{A brief look at XML}
  All of Apertium's input formats are defined using XML.

  XML is a standard (W3C), and it is used everywhere:

  Microsoft Office, OpenOffice, iTunes - all use XML.

  TMX and Formex are two XML-based formats some of you may be familiar with.
\end{frame}

\begin{frame}
  \frametitle{eXtensible Markup Language}

  XML is a markup language - a set of rules for adding annotations to text.

  In XML, these are {\tt $<$tags$>$} and {\tt \&entities$;$}.

\end{frame}

\begin{frame}
  \frametitle{Entities}

  Entities are used to represent other data.

  \begin{exampleblock}{Special characters:}
    \begin{tabular}{ll}
      $<$ & {\tt \&lt;} \\
      $>$ & {\tt \&gt;} \\
      \& & {\tt \&amp;}
    \end{tabular}
  \end{exampleblock}

  Other entities are not used in Apertium, so we will not discuss them.

\end{frame}

\begin{frame}
  \frametitle{Tags /1}

  \begin{exampleblock}{Three kinds of tags:}
    \begin{tabular}{ll}
      Opening tags: & {\tt $<$tag$>$} \\
      Closing tags: & {\tt $<$/tag$>$} \\
      Self-closing (empty) tags: & {\tt $<$tag /$>$}
    \end{tabular}
  \end{exampleblock}

\end{frame}

\begin{frame}
  \frametitle{Tags /2}
  Tags may have attributes: 

  {\tt $<$tag attribute="value"$>$ $<$/tag$>$}

  Tags can contain other tags: 

  {\tt $<$tag$>$ $<$other-tag$>$ $<$/other-tag$>$ $<$/tag$>$}

  \begin{alertblock}{Tags must not overlap!}
    {\tt $<$tag$>$ $<$other-tag$>$ $<$/tag$>$ $<$/other-tag$>$}

   causes a parse error. 
  \end{alertblock}
\end{frame}
\begin{frame}
  \frametitle{A brief look at UNIX}

  \begin{exampleblock}{Example of tagset:}
    \begin{footnotesize}
    \begin{alltt}
      <?xml version="1.0" encoding="UTF-8"?>\\
      <dictionary>\\
      ~<alphabet></alphabet>\\
      ~<sdefs>\\
      ~~...\\
      ~</sdefs>\\
      ~~<pardefs>\\
      ~~~<pardef n="example\_\_n">\\
      ~~~~...\\
      ~~~</pardef>\\
      ~~</pardefs>\\
      ~~<section>\\
      ~~~<e><p><l></l><r></r></p></e>\\
      ~~</section>\\
      </dictionary>\\
    \end{alltt}
    \end{footnotesize}
\end{exampleblock}

\end{frame}

\end{document}
