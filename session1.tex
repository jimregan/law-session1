\documentclass{beamer} %[compress, blue]
\mode<presentation>

\setbeamercolor{frametitle}{fg=lightred,bg=darkred}
\setbeamercolor{title}{fg=lightred,bg=darkred}
\usepackage[spanish,polutonikogreek,english]{babel}
\usepackage{ucs}
\usepackage{times}
\usepackage[T1]{fontenc}
\usepackage{tipa} %ŋ character
\usepackage[utf8x]{inputenc}
\usepackage[absolute,overlay]{textpos}
\usepackage{graphicx}
%\usepackage[small,bf]{caption}
%\usepackage{tabularx}
\usepackage{tikz}
\usepackage{url}
\usepackage{gb4e}
\usepackage{linguex}
\usepackage{cgloss4e}
\usepackage{multirow}
\usepackage{ragged2e} 
\usepackage{wasysym}
\usepackage{alltt}

\usetheme{Warsaw}

\definecolor{darkred}{RGB}{153,0,0}
\definecolor{lightred}{RGB}{226,200,200}
\definecolor{ranis}{RGB}{ 128,128,128}

\newcommand\cyrtext[1]{{\fontencoding{T2A}\selectfont #1}}
\newcommand\grktext[1]{{\fontencoding{LGR}\fontfamily{pmt}\selectfont #1}}

\useoutertheme[subsection=false]{smoothbars}

%\usecolortheme{seahorse} %beetle, albatross, fly, default light, seahorse, crane, dove
\usecolortheme[named= darkred]{structure}

\setbeamertemplate{footline}[text line]{} % makes the footer EMPTY
\setbeamertemplate{headline}[text line]{} % header empty

\setbeamersize{text margin left=0.5cm}
\setbeamertemplate{navigation symbols}{}


%\logo{\includegraphics[height=1.6cm]{logoLAW.png}}
\pgfdeclareimage[height=1.6cm]{logo}{logoLAWnotrans.png}

\setlength{\TPHorizModule}{1mm}
\setlength{\TPVertModule}{1mm}

\newcommand{\MyLogoCentred}{
\begin{textblock}{14}(57.2,10.5)
  \pgfuseimage{logo}
\end{textblock}
}

\newcommand{\MyLogoBottomCentred}{
\begin{textblock}{14}(53.5,70)
  \pgfuseimage{logo}
\end{textblock}
}

\newcommand{\MyLogoBottomRight}{
\begin{textblock}{14}(112.2,80.0)
  \pgfuseimage{logo}
\end{textblock}
}

\newcommand{\MyLogo}{
\begin{textblock}{14}(112.2,0.5)
  \pgfuseimage{logo}
\end{textblock}
}

\date{2nd May 2011}
\title{Session 1: Dictionary basics}

\author{Jimmy O'Regan}

\begin{document}


\frame{\titlepage \MyLogoBottomCentred}


\begin{frame}
  \frametitle{Dictionary Basics: Table of Contents}
  \tableofcontents
\end{frame}


\section{Preamble: XML}

\begin{frame}
  \frametitle{A brief look at XML}
  All of Apertium's input formats are defined using XML.

  XML is a standard (W3C), and it is used everywhere:

  Microsoft Office, OpenOffice, iTunes - all use XML.

  TMX and Formex are two XML-based formats some of you may be familiar with.
\end{frame}

\begin{frame}
  \frametitle{eXtensible Markup Language}

  XML is a markup language - a set of rules for adding annotations to text.

  In XML, these are {\tt $<$tags$>$} and {\tt \&entities$;$}.

\end{frame}

\begin{frame}
  \frametitle{Entities}

  Entities are used to represent other data.

  \begin{exampleblock}{Special characters:}
    \begin{tabular}{ll}
      $<$ & {\tt \&lt;} \\
      $>$ & {\tt \&gt;} \\
      \& & {\tt \&amp;}
    \end{tabular}
  \end{exampleblock}

  Other entities are not used in Apertium, so we will not discuss them.

\end{frame}

\begin{frame}
  \frametitle{Tags /1}

  \begin{exampleblock}{Three kinds of tags:}
    \begin{tabular}{ll}
      Opening tags: & {\tt $<$tag$>$} \\
      Closing tags: & {\tt $<$/tag$>$} \\
      Self-closing (empty) tags: & {\tt $<$tag /$>$}
    \end{tabular}
  \end{exampleblock}

\end{frame}

\begin{frame}
  \frametitle{Tags /2}
  Tags may have attributes: 

  {\tt $<$tag attribute="value"$>$ $<$/tag$>$}

  Tags can contain other tags: 

  {\tt $<$tag$>$ $<$other-tag$>$ $<$/other-tag$>$ $<$/tag$>$}

  \begin{alertblock}{Tags must not overlap!}
    {\tt $<$tag$>$ $<$other-tag$>$ $<$/tag$>$ $<$/other-tag$>$}

   causes a parse error. 
  \end{alertblock}
\end{frame}

\section{Preamble: Some NLP terms}

\begin{frame}
\frametitle{Stem}
A {\it stem} is the main part of a word, to which affixes are added.

\pause

We use a more literal, computing-oriented definition: the stem is the longest
common part of a word.

\end{frame}

\begin{frame}
\frametitle{Morphological analysis}

 \begin{center}
 {\Large \grktext{μορφή}}
 \end{center}

\pause
  \begin{center}
   (form)
  \end{center}

\pause

Analysis of the {\it forms} of words.

\end{frame}

\begin{frame}
\frametitle{Tokenisation}

  \begin{center}
    {\tt |}Splitting{\tt |}text{\tt |},{\tt |}into{\tt |}pieces{\tt |}.{\tt |}
  \end{center}
\end{frame}

\section{Dictionary overview}

\begin{frame}
  \frametitle{Dictionaries}
  
  \begin{block}{Uses of dictionaries}
    \begin{itemize}
    \item Morphological analysis/generation
    \item Bilingual lexicon
    \item Orthographic operations (post-generation)
    \end{itemize}
  \end{block}

\end{frame}

\begin{frame}
  \frametitle{High level overview}

  \begin{exampleblock}{What a dictionary looks like:}
    \begin{footnotesize}
    \begin{alltt}
      <?xml version="1.0" encoding="UTF-8"?>\\
      <dictionary>\\
      ~<alphabet></alphabet>\\
      ~<sdefs>\\
      ~~<sdef n="n" c="noun"/>\\
      ~~...\\
      ~</sdefs>\\
      ~<pardefs>\\
      ~~<pardef n="example\_\_n">\\
      ~~~...\\
      ~~</pardef>\\
      ~</pardefs>\\
      ~<section id="id" type="standard">\\
      ~~...\\
      ~</section>\\
      </dictionary>\\
    \end{alltt}
    \end{footnotesize}
\end{exampleblock}

\end{frame}

\begin{frame}
  \frametitle{alphabet}

  {\tt $<$alphabet$>$} is a slightly misleading name, but
  {\tt $<$characters-to-use-when-tokenising$>$} would be 
  difficult to remember.

  {\tt $<$alphabet$>$} must be present, but may be empty.
\end{frame}

\begin{frame}
  \frametitle{sdefs}

  {\tt $<$sdefs$>$} contains the definitions of symbols used
  in the dictionary.

  {\tt $<$sdefs$>$} may be absent (symbols are not used
  in post generation).
\end{frame}

\begin{frame}
  \frametitle{sdef}

  Each {\tt $<$sdef$>$} defines a symbols.

  The {\tt n} attribute contains the {\it n}ame of the symbol.

  The (optional) {\tt c} attribute contains a {\it c}omment.

\end{frame}

\begin{frame}
  \frametitle{pardefs}

  {\tt $<$pardefs$>$} defines the set of paradigms.

\end{frame}

\begin{frame}
  \frametitle{pardef}

  Each {\tt $<$pardef$>$} defines a paradigm.

  The {\tt n} attribute contains the {\tt n}ame.

  \pause

  More about paradigms later.

\end{frame}

\begin{frame}
  \frametitle{section}

  A {\tt $<$section$>$} contains the entries of the dictionary.

  The {\tt id} attribute should be self-explanatory; each section must 
  also have a tokenisation {\tt type} (usually {\tt standard}). 

\end{frame}

\section{Entries}

\begin{frame}
  \frametitle{A simple entry}

  \begin{exampleblock}{Simple entries:}
    \begin{footnotesize}
    \begin{alltt}
      <e><p><l>beer</l><r>beer<s n="n"/><s n="sg"/></r></p></e> \\
      <e><p><l>beers</l><r>beer<s n="n"/><s n="pl"/></r></p></e>
    \end{alltt}
    \end{footnotesize}
\end{exampleblock}


\end{frame}

\begin{frame}
  \frametitle{A simple entry, take two}

  \begin{exampleblock}{Simple entries:}
    \begin{footnotesize}
    \begin{alltt}
      <e> \\
      ~~<p> \\
      ~~~~<l>beer</l> \\
      ~~~~<r>beer<s n="n"/><s n="sg"/></r> \\
      ~~</p> \\
      </e> \\
      <e> \\
      ~~<p> \\
      ~~~~<l>beers</l> \\
      ~~~~<r>beer<s n="n"/><s n="pl"/></r> \\
      ~~</p> \\
      </e>
    \end{alltt}
    \end{footnotesize}
\end{exampleblock}

\end{frame}

\begin{frame}
  \frametitle{More simple entries}

  \begin{exampleblock}{Simple entries:}
    \begin{footnotesize}
    \begin{alltt}
      <e><p><l>cat</l><r>cat<s n="n"/><s n="sg"/></r></p></e> \\
      <e><p><l>cats</l><r>cat<s n="n"/><s n="pl"/></r></p></e> \\
      <e><p><l>beer</l><r>beer<s n="n"/><s n="sg"/></r></p></e> \\
      <e><p><l>beers</l><r>beer<s n="n"/><s n="pl"/></r></p></e> \\
      <e><p><l>dog</l><r>dog<s n="n"/><s n="sg"/></r></p></e> \\
      <e><p><l>dogs</l><r>dog<s n="n"/><s n="pl"/></r></p></e>
    \end{alltt}
    \end{footnotesize}
\end{exampleblock}

A lot of duplication.
\end{frame}

\begin{frame}
  \frametitle{Paradigms}

  \begin{exampleblock}{Simple entries:}
    \begin{footnotesize}
    \begin{alltt}
      <e> \\
      ~~<p> \\
      ~~~~<l>beer</l> \\
      ~~~~<r>beer<s n="n"/><s n="sg"/></r> \\
      ~~</p> \\
      </e> \\
      <e> \\
      ~~<p> \\
      ~~~~<l>beers</l> \\
      ~~~~<r>beer<s n="n"/><s n="pl"/></r> \\
      ~~</p> \\
      </e>
    \end{alltt}
    \end{footnotesize}
\end{exampleblock}

\end{frame}

\begin{frame}
  \frametitle{Paradigms}

  \begin{exampleblock}{Identify the common part:}
    \begin{footnotesize}
    \begin{alltt}
      <e> \\
      ~~<p> \\
      ~~~~<l>{\bf beer}</l> \\
      ~~~~<r>{\bf beer}<s n="n"/><s n="sg"/></r> \\
      ~~</p> \\
      </e> \\
      <e> \\
      ~~<p> \\
      ~~~~<l>{\bf beer}s</l> \\
      ~~~~<r>{\bf beer}<s n="n"/><s n="pl"/></r> \\
      ~~</p> \\
      </e>
    \end{alltt}
    \end{footnotesize}
\end{exampleblock}
\end{frame}

\begin{frame}
  \frametitle{Paradigms}

  \begin{exampleblock}{Extract the common part:}
    \begin{footnotesize}
    \begin{alltt}
      <e> \\
      ~~<p> \\
      ~~~~<l></l> \\
      ~~~~<r><s n="n"/><s n="sg"/></r> \\
      ~~</p> \\
      </e> \\
      <e> \\
      ~~<p> \\
      ~~~~<l>s</l> \\
      ~~~~<r><s n="n"/><s n="pl"/></r> \\
      ~~</p> \\
      </e>
    \end{alltt}
    \end{footnotesize}
\end{exampleblock}
\end{frame}

\begin{frame}
  \frametitle{Paradigms}

  \begin{exampleblock}{Make a paradigm, and name it:}
    \begin{footnotesize}
    \begin{alltt}
      <pardef n="beer\_\_n"> \\
      ~~<e> \\
      ~~~~<p> \\
      ~~~~~~<l></l> \\
      ~~~~~~<r><s n="n"/><s n="sg"/></r> \\
      ~~~~</p> \\
      ~~</e> \\
      ~~<e> \\
      ~~~~<p> \\
      ~~~~~~<l>s</l> \\
      ~~~~~~<r><s n="n"/><s n="pl"/></r> \\
      ~~~~</p> \\
      ~~</e> \\
      </pardef>
    \end{alltt}
    \end{footnotesize}
\end{exampleblock}
\end{frame}

\begin{frame}
  \frametitle{Using a paradigm}

  \begin{exampleblock}{Examples, using paradigms:}
    \begin{footnotesize}
    \begin{alltt}
      <e><p><l>cat</l><r>cat</r></p><par n="beer\_\_n"/></e> \\
      <e><p><l>beer</l><r>beer</r></p><par n="beer\_\_n"/></e> \\
      <e><p><l>dog</l><r>dog</r></p><par n="beer\_\_n"/></e> \\
    \end{alltt}
    \end{footnotesize}
\end{exampleblock}

\pause
Still some duplication.
\end{frame}

\begin{frame}
  \frametitle{Using a paradigm}

  \begin{exampleblock}{Examples, using paradigms, take 2}
    \begin{footnotesize}
    \begin{alltt}
      <e><i>cat</i><par n="beer\_\_n"/></e> \\
      <e><i>beer</i><par n="beer\_\_n"/></e> \\
      <e><i>dog</i><par n="beer\_\_n"/></e> \\
    \end{alltt}
    \end{footnotesize}
\end{exampleblock}

\end{frame}

\section{Putting it together}
\begin{frame}
  \frametitle{A full dictionary}

  \begin{exampleblock}{What a dictionary looks like, take two:}
    \begin{footnotesize}
    \begin{alltt}
      <?xml version="1.0" encoding="UTF-8"?>\\
      <dictionary>\\
      ~<alphabet>abcdefghijklmnopqrstuvwxyz</alphabet>\\
      ~<sdefs>\\
      ~~<sdef n="n" c="noun"/>\\
      ~~<sdef n="sg" c="singular"/>\\
      ~~<sdef n="pl" c="plural"/>\\
      ~</sdefs>\\
      ~<pardefs>\\
      ~~<pardef n="house\_\_n">\\
      ~~~<e><p><l></l><r><s n="n"/><s n="sg"/></r></p></e> \\
      ~~~<e><p><l>s</l><r><s n="n"/><s n="pl"/></r></p></e> \\
      ~~</pardef>\\
      ~</pardefs>\\
      ~<section id="id" type="standard">\\
      ~~<e><i>cat</i><par n="house\_\_n"/></e> \\
      ~~<e><i>beer</i><par n="house\_\_n"/></e> \\
      ~~<e><i>dog</i><par n="house\_\_n"/></e> \\
      ~</section>\\
      </dictionary>\\
    \end{alltt}
    \end{footnotesize}
\end{exampleblock}
\end{frame}

\begin{frame}
\frametitle{Congratulations}
You now know how to make a basic dictionary.
\end{frame}

\begin{frame}
\frametitle{Compiling the dictionary}

To compile the dictionary:

\$ {\tt lt-comp lr dictionary.xml dictionary.bin}

{\tt id@standard 12 14}

\end{frame}

\begin{frame}
\frametitle{Using the dictionary}

{\tt echo dog|lt-proc dictionary.bin}

{\tt $\^{}$dog/dog$<$n$><$sg$>$\$}
\end{frame}

\end{document}
